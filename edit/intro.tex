\section{Introdução}

Este trabalho foi desenvolvindo para implementar um servidor RADIUS em conjunto com uma
base de dados LDAP para prover um serviço de autenticação de usuário a uma rede.
As seções seguintes abordam a definição do problema, ou seja, a motivação da implementação
de um servidor RADIUS e LDAP, a solução encontrada para a implementação do protocolo, o
desenvolvimento feito até a finalização do trabalho, dificuldades encontradas durante o
desenvolvimento e os resultados encontrados.


\section{O Problema}

\subsection{Radius}

O protocolo Remote Authentication Dial-in User Service (RADIUS) foi desenvolvido pela
Livingston Enterprises Inc. como um servidor de acesso e autenticação. RADIUS é um
protocolo cliente/servidor. O cliente é tipicamente um Network Access Server (NAS) que
pode ser um roteador ou outro servidor de acesso e o servidor é um processo daemon que
pode ser instalado em um sistema UNIX ou Windows NT.

O cliente é responsável por passar as informações do usuário ao servidor RADIUS e tomar
uma ação de acordo com a resposta, que pode ser de permissão ou bloqueio do acesso à rede.
O servidor é responsável por receber as informações do usuário que requisita o acesso,
autentica o usuário, e retorna as informações necessárias para o cliente devolver ao usuário.

\subsubsection{Segurança}

(descobrir como a transmissão dos dados é feita, ou seja, a encriptação dos dados entre
server e client)
O servidor e cliente RADIUS se autenticam por meio de uma chave secreta que nunca é
transmitida pela rede. Além disso, qualquer senha de usuário são encriptadas antes de
serem enviadas do cliente ao servidor, eliminando a possibilidade de roubo de informações
com snoops de rede.

\subsection{LDAP}

\section{Solução e Desenvolvimento}

\section{Dificuldades}

\section{Resultados}

