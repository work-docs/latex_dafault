\section{Introdução}

Para o correto funcionamento de equipamentos conectados a redes, algumas
configurações básicas são necessárias.

\section{Objetivos}

\begin{itemize}
  \item Compreender as configurações básicas para navegabilidade em uma
    rede de computadores.
  \item Exercitar configurações básicas em diferentes sistemas operacionais
    e entender como usar ferramentas de diagnóstico para validar configurações.
\end{itemize}

\section{Realização do Laboratório}

\textbf{Montagem de rede interconectada para o experimento}: uma simples rede
foi criada utilizando um roteador para distribuição da rede com o endereço
194.86.34.*. Um computador foi utilizado para o experimento com o sistema
operacional Debian.

\textbf{Configurar os clientes na rede de testes}: o serviço de configuração
automatida da rede \textbf{network-manager} foi parado para realizar as
configurações manuais.

Foi incluido as novas configurações de rede \textit{eth0} no arquivo \textbf{/etc/network/interfaces}:

\lstset{style=basic}
\begin{lstlisting}
auto eth0
iface eth0 inet static
  address 194.86.34.10
  netmask 255.255.255.0
  gateway 194.86.34.1
\end{lstlisting}

A interface de rede é configurada com a chamada do comando \textbf{ifup eth0}.
Utiliza-se deste arquivo para realizar a configuração permanente no sistema,
ou seja, quando a máquina é reiniciada essas configurações não serão perdidas.
As configurações de rede também podem ser feitas diretamente em pelo comando
\textbf{ifconfig <INTERFACE> <IP> netmask <NETMASK>} e pelo comando
\textbf{route add default gw <GATEWAY-IP>}.

\textbf{Validando as configurações}: utilizando o comando \textbf{ping}, foi possível realizar
a verificação da conexão com os computadores da rede, no caso foi realizado
o ping no próprio roteador. Outro validação foi realizar o ping com algum
computador fora da rede, no caso foi feito um ping no servidor 8.8.8.8 que é
o servidor DNS da Google.

\section{Questões para Estudo}

\begin{enumerate}
  \item \textbf{Há alguma forma mais simples de se realizar a configuração dos
  equipamentos para que sejam devidamente conectados à rede?}

  As aplicações já presentes nos sistemas operacionais podem ser utilizadas
  para realizar a configuração automática das interfaces de rede. Sendo assim
  o ip, gateway, netmask são adquirido por serviço DHCP, bem como a identificação
  dos servidores DNS de resolução de nomes.

  \item \textbf{Qual é a lista mínima de informações necessárias para que
  determinado equipamento fique plenamente operacional em uma rede?}

  As principais configurações necessárias para que dois computadores se
  comuniquem dentro de uma rede são: ip, netmask e gateway. O ip é o endereço
  que identifica um dispositivo na rede. O netmask diz qual é a rede em que
  o dispositívo está inserido, ou seja, no nosso experimento, a netmask
  255.255.255.0 diz aos dispositivos que a rede é composta de ips de
  194.86.34.0 até 194.86.34.254. O gateway é o endereço da saída da
  rede, ou seja, o roteador usado para se comunicar com outros computadores
  fora da rede.

  \item \textbf{O que acontece quando alguma das informações necessárias é
  suprimida? Elabore melhor os cenários.}

  Elaborando cenários onde há a falta de cada informação.
  Falta de ip: o dispositivo não pode ser identificado dentro
  da rede interna ou externa, pois o roteador ou switch não sabe de onde ou
  para onde enviar os pacotes.
  Falta de netmask: o dispositivo considera que é possível se comunicar com
  todos os IPs disponíveis, ou seja, a mascara é considerado como 0.0.0.0.
  Falta de gateway: o dispositivo não consegue se comunicar com os dispositívos
  fora da rede.

\end{enumerate}


