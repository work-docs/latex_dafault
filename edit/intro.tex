\section{Introdução}

Uma vez devidamente configurado, um equipamento que se comunica através de uma rede de
computadores como a Internet se faz útil ao interagir com outros equipamentos para o provimento
de serviços a usuários.

\section{Objetivos}

\begin{itemize}
  \item Exercitar as configurações básicas para navegabilidade em uma rede de computadores bem como
    como usar ferramentas de diagnóstico para validar configurações;
  \item Exercitar uma comunicação típica HTTP por meio de ferramenta de diagnóstico (telnet);
  \item Exercitar o início de uma comunicação típica SMTP por meio de ferramenta de diagnóstico
    (telnet);
  \item Exercitar as configurações de rede, especialmente no que tange ao serviço de resolução de
    nomes.
\end{itemize}

\section{Realização do Laboratório}

\textbf{Montagem de rede interconectada para o experimento}: foi feito a montagem de uma rede WAN utilizando
um \textit{router wireless} com máscara 255.255.255.0 e endereços entre 192.168.10.2 à 192.168.10.254.

\textbf{Configurar os clientes na rede de testes e validar as configurações}: os procedimentos realidados no
laboratório 00 fora refeitos nesta etápa deste laboratório.

\textbf{Interações com o serviço de resolução de nomes (DNS)}:

\textbf{Interações com o serviço HTTP}

\textbf{Interações com o serviço SMTP}

\textbf{DESAFIO: Usando os conceitos explorados na aula sobre Camada de Aplicação, use o comando
telnet para enviar um e-mail a um usuário de determinado servidor de e-mail.}

\begin{enumerate}
  \item \textbf{Em relação ao serviço de resolução de nomes, há um parmetro opcional a ser indicado ao
    comando nslookup. Em que contexto é conveniente indicar um valor para esse
  parmetro?}

\item \textbf{Qual é o papel do arquivo /etc/hosts no processo de resolução de nomes?}

  \item \textbf{Em relação às interações com o protocolo HTTP, foi possível identificar o cabeçalho de uma
    requisição típica? Em relação às respostas do servidor, identifique os campos típicos da
  resposta incluindo descrições sobre as linhas de cabeçalho e o campo de payload.}

  \item \textbf{Em vários dos protocolos ora estudados, foi presenciada uma etapa de autorização que
    preparava uma sessão para a recepção de comandos de determinado cliente. O SMTP
    demonstrou-se um protocolo que não demanda uma etapa de autorização. Em que momento
    isso acontece? O fato de essa etapa ser suprimida resulta em algum risco para um serviço de
  e-mail?}
\end{enumerate}
