\section{Introdução}

Este trabalho foi desenvolvindo para implementar um servidor RADIUS em conjunto com uma
base de dados LDAP para prover um serviço de autenticação de usuário a uma rede.
As seções seguintes abordam a definição do problema, ou seja, a motivação da implementação
de um servidor RADIUS e LDAP, a solução encontrada para a implementação do protocolo, o
desenvolvimento feito até a finalização do trabalho, dificuldades encontradas durante o
desenvolvimento e os resultados encontrados.

Como foi visto em sala de aula, apesar de uma encriptação de chave compartilhada
ser melhor que não ter encriptação alguma, algumas falhas de segurança podem ser
observadas.


\subsection{O Problema}

Em um contexto de uma empresa, a distribuição de internet por Wireless é um ponto de
vulnerabilidade para a rede local. Na maior parte dos casos, uma rede Wifi é protegida
utilizando o protocolo WPA2 Pre-Shared Key (PSK), que consiste na proteção do acesso a internet com uma
chave secreta compartilhada. Isso significa que todos os usuários daquela rede terão acesso
sua conexão criptografada com a mesma chave, que não pode ser revogada. \cite{freeradius}

Especialmente num ambiente corporativo, onde se tem alta rotatividade de pessoas,
o PSK não é o modelo mais seguro de conexão com a internet. Um usuário que não tem
mais vínculo com a empresa não deveria ter mais acesso à rede e o único jeito de se
fazer isto com o PSK é mudando a chave de acesso, que causa um grande \textir{overhead}
para os usuários e não garante muito mais segurança, já que um atacante pode realizar
ataques offline para descobrir a chave \cite{freeradius}

Para resolver este problema, a IEEE, ao descrever os padrões para internet sem fio,
previu um modelo "Empresarial" que difere do PSK no sentido de usar autenticação
por usuário e por sessão \cite{freeradius}

\section{Radius}

O protocolo Remote Authentication Dial-in User Service (RADIUS) foi desenvolvido pela
Livingston Enterprises Inc. como um servidor de acesso e autenticação. RADIUS é um
protocolo cliente/servidor. O cliente é tipicamente um Network Access Server (NAS) que
pode ser um roteador ou outro servidor de acesso e o servidor é um processo daemon que
pode ser instalado em um sistema UNIX ou Windows NT.

O cliente é responsável por passar as informações do usuário ao servidor RADIUS e tomar
uma ação de acordo com a resposta, que pode ser de permissão ou bloqueio do acesso à rede.
O servidor é responsável por receber as informações do usuário que requisita o acesso,
autentica o usuário, e retorna as informações necessárias para o cliente devolver ao usuário.

\subsection{Segurança}

(descobrir como a transmissão dos dados é feita, ou seja, a encriptação dos dados entre
server e client)
O servidor e cliente RADIUS se autenticam por meio de uma chave secreta que nunca é
transmitida pela rede. Além disso, qualquer senha de usuário são encriptadas antes de
serem enviadas do cliente ao servidor, eliminando a possibilidade de roubo de informações
com snoops de rede.

\section{LDAP}

O protocolo Lightweight Directory Access Protocol (LDAP) foi desenvolvido especificamente
para simples leituras/escritas para o X.500 Directory, e é um complemento pra o Directory
Access Protocol (DAP). O X.500 Directory é uma série de padrões de rede que aborda serviço
de diretório. O LDAP utiliza-se do TCP para o transporte de diretórios pela rede e os dados
são codificados como strings (Distinguished Names).

O modelo da X.500 de 1993 é implementado no LDAP e consiste em:
\begin{itemize}
  \item Uma entrada contem vários atributos;
  \item Um atributo tem um nome e um ou mais valores. Os atributos são definidos nos schemas;
  \item Cada entrada tem um identificador único: Distinguished Name (DN). Consiste de um
    Relative Distinguished Name (RDN), construído de alguns atributos seguidos do DN pai.
    O DN pode ser comparado com um caminho de um arquivo e o RDN o nome relativo do arquivo.
    (temos /foo/bar/file.txt como DN e file.txt como RDN).
\end{itemize}

\section{Solução e Desenvolvimento}

\section{Dificuldades}

\section{Resultados}

\bibliography{edit/refs}