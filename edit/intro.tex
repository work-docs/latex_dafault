\section{Introdução}

Uma vez devidamente configurado, um equipamento que se comunica através de uma rede de
computadores como a Internet se faz útil ao interagir com outros equipamentos para o provimento
de serviços a usuários.

Ao longo dos encontros em que foi discutida a Camada de Rede, foram introduzidos vários
protocolos que tipicamente dão suporte às aplicações em rede, que, por sua vez, provêem os
serviços que usamos diariamente.

Discutiu-se o conjunto de protocolos que cooperam no provimento dos serviços típicos de camada
de rede podem ser separados em duas categorias: protocolos que geram tabelas de encaminhamento,
os protocolos de roteamento, e os protocolos que utilizam as tabelas de encaminhamento.

Antes de serem introduzidas ferramentas de inspeção de protocolos de diferentes camadas, faz-se
necessário apresentar ao estudante um conjunto mínimo de ferramentas que permitirão a execução
de um diagnóstico preciso ao se encarar uma situação de interrupção ou instabilidade de serviço
típico de camada de rede.

\section{Objetivos}

\begin{itemize}
  \item  Exercitar as configurações básicas para navegabilidade em uma rede de computadores bem como
    como usar ferramentas de diagnóstico para validar configurações
  \item Exercitar os princípios básicos de uma comunicação em redes TCP/IP, com ênfase nos serviços
    típicos de camada de rede. Conhecer e manipular ferramentas de diagnóstico (ping,
    traceroute, netstat e route) para fixação de conceitos de camada de rede.
\end{itemize}

\section{Questões para Estudo}

\begin{enumerate}
  \item \textbf{A porta de um servidor que provê aplicações sobre TCP pode se encontrar em diferentes
    estados. Quais são esses estados e como evolui a comunicação entre um cliente e um
  servidor TCP quando a porta se apresenta em cada um dos estados possíveis?}

  Quando uma porta é aberta manualmente, ela fica no estado LISTEN, esperando conexoes.
  Quando um socket precisa de uma conexao, ele manda a flag SYN e fica com o estado
  SYN\_SENT (este estado nao foi percebido pela dupla, supoe-se que por causa da velocidade da
  conexao). O socket receptor fica no estado de LISTEN aguardando a conexão do cliente.
  O proximo estado sera ESTABLISHED, se o socket receptor aceitou a conexao.
  O estado FIN\_WAIT1 indica que o client deseja finalizar a conexão. O próximo estado é o 
  TIME\_WAIT indica a espera do ACK do servidor. O servidor, quando recebe o FIN, ele entra em
  estado de CLOSE\_WAIT e responde com o ACK. O cliente recebe esse ACK e entra em estado de
  FIN\_WAIT2. O servidor então manda seu FIN e entra em estado de LAST\_ACK. Assim que o servidor
  recebe o ACK do cliente ele entra em estado CLOSE fechando a conexão.

  \item \textbf{Que tipo de informações o arquivo /etc/services provê?}

  Mapeamento de serviço de internet para as portas e protocolos. Mesmo que o serviço não utiliza
  protocolo UDP ele é mapeado para os dois protocolos TCP e UDP.

  \item \textbf{Que tipo de ferramentas você recomendaria para a repetição dos mesmos procedimentos
    sobre UDP?}

  Utilizando as mesmas ferramentas, mas com a utilização de flags específicas para operações com o UDP

  \item \textbf{Em relação à Etapa 4 do presente roteiro, descreva o que se observou durante as interações
    com o socket servidor.}

  Observou-se que a mensagem enviada pelo telnet aparecia na tela do servidor nc e o inverso também
  era válido. Quando a conexão era encerrada tanto no telnet quanto no nc, a conexão era encerradas e as
  portas fechadas.

  \item \textbf{Como você implementaria um transmissor básico de arquivos usando apenas as ferramentas
    executadas nesse experimento?}

  Considerando uma transferencia de arquivo text ASCII o procedimento seria feito com redirecionameto
  de arquivos. Exemplo:

  \lstset{style=basic}
  \begin{lstlisting}
  # Servidor
  nc -l 8080 > output.txt

  # Cliente
  telnet localhost 8080 < input.txt

  \end{lstlisting}
  \item \textbf{O nmap é considerado uma ferramenta extremamente poderosa. Como você a utilizaria para
    identificar quais são os equipamentos que estão ativos e em execução em uma rede?}

  Utilizando o nmap com a flag -sn que não escaneia as portas. Exemplo:

  \lstset{style=basic}
  \begin{lstlisting}
  nmap -sn 192.168.0.20/24
  \end{lstlisting}
  
  A notação /24 significa que a rede será escaneada entre 192.168.0.0-254 ips.
\end{enumerate}
