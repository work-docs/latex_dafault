\section{Introdução}

Uma vez devidamente configurado, um equipamento que se comunica através de uma rede de
computadores como a Internet se faz útil ao interagir com outros equipamentos para o provimento
de serviços a usuários.

Ao longo dos encontros em que foi discutida a Camada de Rede, foram introduzidos vários
protocolos que tipicamente dão suporte às aplicações em rede, que, por sua vez, provêem os
serviços que usamos diariamente.

Discutiu-se o conjunto de protocolos que cooperam no provimento dos serviços típicos de camada
de rede podem ser separados em duas categorias: protocolos que geram tabelas de encaminhamento,
os protocolos de roteamento, e os protocolos que utilizam as tabelas de encaminhamento.

Antes de serem introduzidas ferramentas de inspeção de protocolos de diferentes camadas, faz-se
necessário apresentar ao estudante um conjunto mínimo de ferramentas que permitirão a execução
de um diagnóstico preciso ao se encarar uma situação de interrupção ou instabilidade de serviço
típico de camada de rede.

\section{Objetivos}

\begin{itemize}
  \item Exercitar as configurações básicas para navegabilidade em uma rede de computadores bem como
    como usar ferramentas de diagnóstico para validar configurações;
  \item Exercitar os princípios básicos de uma comunicação em redes TCP/IP, com ênfase nos serviços
    típicos de camada de rede. Conhecer e manipular ferramentas de diagnóstico (ping,
    traceroute, netstat e route) para fixação de conceitos de camada de rede.
\end{itemize}

\section{Questões para Estudo}

\begin{enumerate}
  \item \textbf{A descrição de informações específicas sobre a interface de rede obtida através da execução
    do comando ifconfig é bem extensa. Partindo do resultado da execução do comando em seu
    equipamento e tomando como referência a interface que foi usada nos testes, descreva em
    detalhes o significado de cada um dos itens discriminados na segunda coluna de saída do
    comando.}

  \item \textbf{Equipamentos que implementam a pilha de protocolos TCP/IP apresentam, tipicamente, uma
    saída padrão durante a execução do comando ifconfig. Indique que saída é essa e como ela
    se faz útil na construção de sistemas que funcionarão sobre redes de comunicações.}

  \item \textbf{A execução do comando ping em uma rede é uma das primeiras medidas para verificação de
    continuidade de serviço. Embora seja um comando muito simples de invocar e cujos
    resultados práticos são muito fáceis de interpretar, a riqueza de detalhes da saída típica de
    execução do ping é grande. Descreva cada um dos itens que compõem a saída de execução
    típica de um comando ping.}

  \item \textbf{Em relação à Etapa 5 do presente roteiro, descreva cada um dos itens que compõem a saída
    de execução típica de um comando netstat -nr. Qual seria a sintaxe usada para extrair a
    mesma informação, porém a partir do comando ip?}

  \item \textbf{Em relação à Etapa 5 do presente roteiro, descreva cada um dos itens que compõem a saída
    de execução típica de um comando netstat -nr. Qual seria a sintaxe usada para extrair a
    mesma informação, porém a partir do comando ip?}

  \item \textbf{Em relação à Etapa 6 do presente roteiro, descreva o processo típico de detecção da rota
    percorrida por um pacote através da execução do comando traceroute}

\end{enumerate}
