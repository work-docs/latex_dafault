\section{Introdução}
Uma vez devidamente configurado, um equipamento que se comunica através de uma rede de
computadores como a Internet se faz útil ao interagir com outros equipamentos para o provimento
de serviços a usuários.

Ao longo dos encontros, foi possível obter uma visão geral de como funcionam redes de comutação
de pacotes sobre os protocolos TCP/IP. Discutiu-se o conjunto de protocolos que cooperam no
provimento dos serviços típicos de camada básica componente da arquitetura de redes TCP/IP:
iniciamos o estudo de protocolos de camada de aplicação, seguindo para a camada de transporte,
depois camada rede e, por fim, camadas de enlace e física.

Agora que os estudantes já tem um conceito consolidado da arquitetura típica das redes TCP/IP,
convém apresentarmos as ferramentas mais especializadas para a depuração de problemas
enfrentados seja na sustentação de uma infraestrutura de redes de computadores, seja na sustentação
de aplicações distribuídas.

\section{Objetivos}

\begin{itemize}
  \item Exercitar as configurações básicas para navegabilidade em uma rede de computadores bem como
    como usar ferramentas de diagnóstico para validar configurações
  \item Exercitar os princípios básicos de uma comunicação em redes TCP/IP, nesse momento com
    ênfase na visão integral da interação entre os protocolos de várias camadas que são coordenados
    para a sustentação de uma aplicação distribuída ou mesmo no provimento de uma infraestrutura
    típica de redes de computadores.
  \item Conhecer e manipular ferramentas de diagnóstico (ping, traceroute, netstat,
    tcpdump e wireshark) para fixação de conceitos de fundamentos de redes de
    computadores.
  \item Identificar os campos típicos dos pacotes de vários protocolos que interagem no provimento de
    aplicação distribuída
\end{itemize}

\section{Questões para Estudo}

\begin{enumerate}
  \item \textbf{Realize a captura de uma sessão HTTP de busca por uma página da Internet. Partindo da
    captura, descreva detalhadamente as etapas envolvidas na requisição de um objeto
    index.html de uma página provida sobre o protocolo HTTP, explicando criteriosamente o
    conteúdo das mensagens. Não se esqueça de dizer a qual das camadas pertencem os
    protocolos envolvidos.}

  
\end{enumerate}
