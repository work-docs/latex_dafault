\section{Introdução}

\section{Descrição do Problema}

\section{\textit{Checklist} de Requisitos}

A lista abaixo contem os requisitos espeficicados para o laboratório e o
seu status de implementação da solução proposta.

\begin{itemize}
  \item (  ) Estudo sobre Vírus: tipos de vírus e vermes, como eles se disseminam;
  \item (  ) Estudo sobre \textit{Device Drivers}: funcionamento e como construir;
  \item (  ) Tutorial da construção do \textit{Device Drivers} (Anexo A);
  \item (  ) Propor um \textit{Device Drive} com duas forma de funcionamento: normal e anormal;
  \item (  ) O \textit{Device Drive} deve ser gerenciado por lsmdo, insmod, rmmod;
  \item (  ) O \textit{Device Drive} deve apresentar na tela o que está ocorrendo;
  \item (  ) Usar processos em ambiente Linux/Linguagem C;
  \item (  ) Executar o programa várias vezes e criar um quadro adequado para apresetar os resultados;
\end{itemize}

\section{Metodologia}

\section{Descrição da Solução}
%TODO: deve conter algoritmos (partes relevantes) README descrevendo a utilização

\section{Conclusão}
%TODO: opniões sobre o projetom
%   Dificuldades
%   Lições aprendidas

%TODO: Anexo contendo:
%   Descrição sobre as funções de manipulação de devices drivers: estrutura de dados e exemplos
%   Apresentar outra estrutura da soloção implementada

%TODO: Anexo contendo:
%   Descrição sobre tipos de virus: virus de setor de boot e virus de macro
%   Alternativas de proteção
