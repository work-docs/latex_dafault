\section{Introdução}

Uma vez devidamente configurado, um equipamento que se comunica através de uma rede de
computadores como a Internet se faz útil ao interagir com outros equipamentos para o provimento
de serviços a usuários.

\section{Objetivos}

\begin{itemize}
  \item Exercitar as configurações básicas para navegabilidade em uma rede de computadores bem como
    como usar ferramentas de diagnóstico para validar configurações;
  \item Exercitar uma comunicação típica HTTP por meio de ferramenta de diagnóstico (telnet);
  \item Exercitar o início de uma comunicação típica SMTP por meio de ferramenta de diagnóstico
    (telnet);
  \item Exercitar as configurações de rede, especialmente no que tange ao serviço de resolução de
    nomes.
\end{itemize}

\section{Realização do Laboratório}
\textbf{Montagem de rede interconectada para o experimento}: foi feito a montagem de uma rede WAN utilizando
um \textit{router wireless} com máscara 255.255.255.0 e endereços entre 192.168.10.2 à 192.168.10.254.
\bigbreak
\textbf{Configurar os clientes na rede de testes e validar as configurações}: os procedimentos realizados no
laboratório 00 não foram refeitos. Foi utilizado a aplicação de configuração automática - network manager -
que realiza a configuração de ip, netmask e gateway.
\bigbreak
\textbf{Interações com o serviço de resolução de nomes (DNS)}: utilizando a aplicação `nslookup` - query para
domain name servers - verificou-se os DNS que proveem os domínios: \textbf{google.com, fga.unb.br, unb.br e localhost}.
\bigbreak

Os resultados para cada um dos comandos:
\lstset{style=basic}
\begin{lstlisting}
$ nslookup google.com
Server:         192.168.10.1
Address:        192.168.10.1#53

Non-authoritative answer:
Name:   google.com
Address: 216.58.222.110

$ nslookup fga.unb.br
Server:         192.168.10.1
Address:        192.168.10.1#53

Non-authoritative answer:
Name:   fga.unb.br
Address: 164.41.94.194

$ nslookup unb.br
Server:         192.168.10.1
Address:        192.168.10.1#53

Non-authoritative answer:
Name:   unb.br
Address: 164.41.101.33

$ nslookup localhost
Server:         192.168.10.1
Address:        192.168.10.1#53

Name:   localhost.unb.br
Address: 127.0.0.1
\end{lstlisting}
\bigbreak
A análise do output do comando temos: o servidor DNS (nome e IP), que é configurado no sistema operacional, no caso
o server mostrado é o ip do router; o nome e IP do servidor procurado utilizado o domain name.
Com a utilização de um parametro no comando, foi testado a especificação de um servidor que não é o
default do sistema operacional. É possível recuperar o servidor de DNS authoritative do host com o comando
'host -t ns host\_name'.
\bigbreak
\lstset{style=basic}
\begin{lstlisting}
$ host -t ns unb.br
unb.br name server dns1.unb.br.
unb.br name server dns3.unb.br.
unb.br name server dns2.unb.br.
unb.br name server dns10.unb.br.

$ nslookup unb.br dns1.unb.br
Server:         dns1.unb.br
Address:        164.41.101.3#53

Name:   unb.br
Address: 164.41.101.33

\end{lstlisting}

\bigbreak
\textbf{Interações com o serviço HTTP}: foi realizado os testes com 'telnet' para a comunicação
HTTP com os servidores: unb.br e localhost. O servidor localhost foi instalado um servidor
local provido pela aplicação Nginx.

Com a utilização do método GET, foi recuperado as páginas requisitadas com a montagem de um
cabeçario mínimo HTTP especificando apenas o host.
\bigbreak
\lstset{style=basic}
\begin{lstlisting}
$ telnet www.unb.br http
Trying 164.41.101.33...
Connected to www.unb.br.
Escape character is '^]'.
GET / HTTP/1.1
Host: www.unb.br

HTTP/1.1 200 OK
Date: Thu, 14 Sep 2016 12:19:20 GMT
Server: Apache
Set-Cookie: CAKEPHP=n9nkcnsqcps8l9811cvli4dgd4; expires=Sun, 15 Sep 2041 18:19:45 GMT; path=/
P3P: CP="NOI ADM DEV PSAi COM NAV OUR OTRo STP IND DEM"
Set-Cookie: CAKEPHP=n9nkcnsqcps8l9811cvli4dgd4; expires=Sun, 15 Sep 2041 18:19:45 GMT; path=/
Set-Cookie: CAKEPHP=n9nkcnsqcps8l9811cvli4dgd4; expires=Sun, 15 Sep 2041 18:19:45 GMT; path=/
Set-Cookie: CAKEPHP=n9nkcnsqcps8l9811cvli4dgd4; expires=Sun, 15 Sep 2041 18:19:45 GMT; path=/
Set-Cookie: CAKEPHP=n9nkcnsqcps8l9811cvli4dgd4; expires=Sun, 15 Sep 2041 18:19:45 GMT; path=/
Set-Cookie: CAKEPHP=n9nkcnsqcps8l9811cvli4dgd4; expires=Sun, 15 Sep 2041 18:19:45 GMT; path=/
Connection: close
Transfer-Encoding: chunked
Content-Type: text/html

9626
  <!DOCTYPE html PUBLIC "-//W3C//DTD XHTML 1.0 Transitional//EN" "http://www.w3.org/TR/xhtml1/DTD/xhtml1-transitional.dtd">
  <html lang="pt-BR" xmlns="http://www.w3.org/1999/xhtml" xml:lang="pt-BR">
  <head>
...


$ telnet localhost http
Trying ::1...
Connected to localhost.
Escape character is '^]'.
GET / HTTP/1.1
Host:localhost
Connection:close

HTTP/1.1 200 OK
Server: nginx/1.6.2
Date: Thu, 15 Sep 2016 12:31:35 GMT
Content-Type: text/html
Content-Length: 867
Last-Modified: Wed, 14 Sep 2016 23:45:29 GMT
Connection: close
ETag: "57d9e119-363"
Accept-Ranges: bytes

<!DOCTYPE html>
<html>
...
\end{lstlisting}

\textbf{Interações com o serviço SMTP + DESAFIO}: foi feito uma conexão com o servidor de SMTP local
configurado utilizando a ferramenta Postfix. Seguindo os passo do roteiro e outras pesquisas e exemplos
na internet, foi feito o envio de uma email para uma conta registrada no domínio @gmail.com.


\begin{lstlisting}
$ telnet localhost smtp
Trying ::1...
Connected to localhost.
Escape character is '^]'.
220 jessie.raw ESMTP Postfix (Debian/GNU)
HELO localhost
250 jessie.raw
MAIL FROM:<root@localhost.com>
250 2.1.0 Ok
RCPT TO:<paulohtfs@gmail.com>
250 2.1.5 Ok
DATA
354 End data with <CR><LF>.<CR><LF>
From: sender@example.com
To: recipient@example.com
Subject: Test message

Hello there!
.
250 2.0.0 Ok: queued as 0EDEB204F9
quit
221 2.0.0 Bye
Connection closed by foreign host.
\end{lstlisting}

\begin{enumerate}
  \item \textbf{Em relação ao serviço de resolução de nomes, há um parametro opcional a ser indicado ao
    comando nslookup. Em que contexto é conveniente indicar um valor para esse
    parametro?}

    Quando já se sabe o servidor de DNS authoritative que possui o IP do hostname desejado, isso
    evita a etapa de busca utilizando os DNS Root e DNS TLD diminuindo o tempo de resposta.

  \item \textbf{Qual é o papel do arquivo /etc/hosts no processo de resolução de nomes?}

    O arquivo /etc/hosts faz o mapeamento do hostname e IP. Ele é cosultado antes da resolução de
    nomes do servidor DNS mais próximo.

  \item \textbf{Em relação às interações com o protocolo HTTP, foi possível identificar o cabeçalho de uma
    requisição típica? Em relação às respostas do servidor, identifique os campos típicos da
    resposta incluindo descrições sobre as linhas de cabeçalho e o campo de payload.}

  Sim, é possível especificar quais os campos e valores do cabeçario de uma requisição que pode conter Host,
  Connection, etc. Os campos de resposta são: versão, codigo da resposta,
  menssagem da resposta, server application, data da resposta, tipo de conteúdo, comprimento do conteúdo
  opção de controle e o payload, ou seja, o conteúdo da página.

  \item \textbf{Em vários dos protocolos ora estudados, foi presenciada uma etapa de autorização que
    preparava uma sessão para a recepção de comandos de determinado cliente. O SMTP
    demonstrou-se um protocolo que não demanda uma etapa de autorização. Em que momento
    isso acontece? O fato de essa etapa ser suprimida resulta em algum risco para um serviço de
    e-mail?}

    Não ocorre a etapa de autorização, contando que tenha um servidor de e-mail. Pode existir os riscos da
    interceptação da mensagem enviada e ocorrencia de spam, mas não há recuperação de dados que estão
    presentes no servidor de e-mail.

\end{enumerate}
